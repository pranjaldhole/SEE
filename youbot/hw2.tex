\chapter{youBot placing experiment}
\section{Deliverables 6.2}
\textbf{Update your previous week’s report and include the following:}

\subsection{If necessary, rerun your experiment according to the feedback in class and describe how the new run improves on the previous run.}
We were not asked to rerun the experiment.

\subsection{Find suitable statistical techniques for analyzing the experimental results (using what you’ve learned during the LEGO experiment) and answer the following questions:}
\subsubsection{Knowing the ground-truth values of the expected final object poses, what is the accuracy of the arm? What about its precision?}

\subsubsection{Does the distribution of the final object poses follow a Gaussian distribution?}

\subsubsection{How much does changing the shape and mass of an object affect the final placing pose? Is this effect statistically significant? If the effects are indeed significant, what might have caused them? Hint: Statistical significance can only be determined by performing appropriate hypothesis tests.}

\subsubsection{If we are to use a single camera measurement for determining the final object pose instead of multiple filtered measurements, is the effect on the final pose estimates significant?}

\subsection{Mention the list of used software and include the source of any functions you wrote for performing your analysis.}
\begin{itemize}
\item We output generated by \texttt{Subscriber.py} was a broken \texttt{json} file. Therefore importing the data was non-trivial. We wrote a \texttt{runscripts.sh} bash script that was a workaround to search for the relevant data from all the recordings and then write it to \texttt{csv} files.
\item The \texttt{runscripts.sh} requires \texttt{read.py} as dependency file.
\item Each data point was collection of 50 poses $(x, y, \theta)$. This was done to add redundancy such that the measurement is tolerant to measurement-faults. The \texttt{runscripts.sh} file already take the mean value of these readings and writes the mean value to respective \texttt{csv} file.
\item All the data stored in \texttt{csv} files is then imported to a \texttt{jupyter notebook} for plotting and statistical analysis.
\item All the code is attached alongside with the submitted report.
\end{itemize}

%%%%%%%%%%%%%%%%%%%%%%%%%%%%%%%%%%%%%%%%%%%%%%%%%%%%%%%%%%%%%%%%%%%%%%%%%%%%%