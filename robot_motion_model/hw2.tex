\section*{Assignment 1.2}
\subsection*{1. Robot configuration}
\label{robot-config}
\subsection*{2. Deliverables 1.2}
\subsubsection*{2.1 Robot design}
\textbf{Materials}
\begin{wrapfigure}{r}{0.4\textwidth}
  \vspace{-50pt}
  \begin{center}
    \includegraphics[width=0.39\textwidth]{fringe.jpeg}
  \end{center}
  \vspace{-20pt}
  \caption{Fringes formed on paper by LED lights.}
\end{wrapfigure}

We used the provided Lego Mindstorms 9797 toolkit for the construction of our robot.\\
The robot design is shown in the previous section.\\
For measurement purposes, we have mounted two LED lights with an approx. distance of $9.5 \pm 0.2$ cm in front of the Lego bot. \\
We have attached two resistors of 330 Ohms to each of the LEDs in order to reduce the current flowing through the LEDs. \\
The uncertainty is due to imprecise ruler measurements on each LED. The LED lights formed nice fringes on paper which made marking the center quite easy.\\
We use the provided  2.5x2.5cm grid paper for marking the positions of robot before and after each motion.\\
The corners of the grid paper were glued to the table in order to prevent an movement due to robot motion. 
\subsubsection*{2.2 Experiment}
\begin{itemize}
\item We run the LEGO NXT differential drive robot in three different configurations, viz., straight line motion, right turn motion and left turn motion.
\item The starting position of LEDs is marked on the grid paper and for each trial we align the centres of LED lights on the initial marked position on grid paper.
\item We collected 20 trial data of end robot positions for each configuration.
\item The observations made are tabulated in \texttt{motionData.csv} file attached.
\item The initial position for each configuration run is tabulated in \texttt{motionData\_init.csv} file.
\item In some of the trials the motion initiated in jerks which resulted in larger deviation. These points can be seen as extreme points from the mean of the distribution in each wheel data.
\item Another source of error is the orientation of the back wheel in the initial position. Since it is a passive castor wheel, this could have introduced some discrepancy in the motion.

\begin{figure}[H]
\centering
\begin{subfigure}[b]{0.49\textwidth}
\includegraphics[width=\textwidth]{Left-turn.png}
\caption{Left turn motion}
\label{left}
\end{subfigure}
\begin{subfigure}[b]{0.49\textwidth}
\includegraphics[width=\textwidth]{Right-turn.png}
\caption{Right turn motion}
\label{right}
\end{subfigure}
\begin{subfigure}[b]{0.49\textwidth}
\includegraphics[width=\textwidth]{Straight.png}
\caption{Straight line motion}
\label{straight}
\end{subfigure}
\caption{Robot motion data. {\color{green}{x/y with units defined.}}}
\label{data}
\end{figure}

\item In fig. \ref{data}, the robot motion data is visualized. The initial positions are marked by `+' sign and the mean positions are marked by `x' sign.
\item Linear Deviation of right wheel : 
	$$\delta x_{wR} = x_{wR}' - x_{wR} \qquad \delta y_{wR} = y_{wR}' - y_{wR}$$
\item Linear Deviation of left wheel : 
	$$\delta x_{wL} = x_{wL}' - x_{wL} \qquad \delta y_{wL} = y_{wL}' - y_{wL}$$
\item Angular Deviation (Angle between two lines joining expected wheel position and experimental wheel positions) : 
	$$\delta \theta_i = tan^{-1}\frac{\delta x_{i}}{\delta y_{i}}$$
\item \textbf{Random errors}: Random errors are caused by inability of the sensors to measure the input beyond certain resolution. In our case inability to resolve current supplied beyond certain resolution would result in slight deviations in torques torques generated in motor which results in wheel rotation and ultimately results in variations in distance travelled by robot. We can correct this error by repeating the experiment multiple number of times and taking the mean value of the measurements. The mean value of change of pose can be given as,
\begin{align*}
\overline{\delta x} = \frac{1}{20} \sum_{i=1}^{20} \delta x_i \\
\overline{\delta y} = \frac{1}{20} \sum_{i=1}^{20} \delta y_i \\
\overline{\delta \theta} = \frac{1}{20} \sum_{i=1}^{20} \delta \theta_i
\end{align*}
The standard error of the mean for each variable is given by,
\begin{align*}
\hat{\sigma}_{\overline{\delta x}} = \sqrt{\frac{\sum_{i=1}^{20} (\delta x - \overline{\delta x})^2}{19 * 20}} \\
\hat{\sigma}_{\overline{\delta y}} = \sqrt{\frac{\sum_{i=1}^{20} (\delta y - \overline{\delta y})^2}{19 * 20}} \\
\hat{\sigma}_{\overline{\delta \theta}} = \sqrt{\frac{\sum_{i=1}^{20} (\delta \theta - \overline{\delta \theta})^2}{19 * 20}}
\end{align*}
\item The uncertainty in pose due to random errors for each trial can be given as,
\begin{align*}
\delta x = \delta x_{\text{measured}} \pm \hat{\sigma}_{\overline{\delta x}} \\
\delta y = \delta y_{\text{measured}} \pm \hat{\sigma}_{\overline{\delta y}} \\
\delta \theta = \delta \theta_{\text{measured}} \pm \hat{\sigma}_{\overline{\delta \theta}}
\end{align*}
\item \textbf{Measurement error}: The accuracy of the position marker measurement due to unresolved LED center resolution was $\pm 1mm$.
\begin{table}[ht]
\centering
\begin{tabular}{| l | c | c | c | c | c | c |}
\hline
 & \multicolumn{2}{c|}{left-turn motion} & \multicolumn{2}{c|}{ right-turn motion} & \multicolumn{2}{c|}{straight motion} \\
\hline
 & right wheel & left wheel & right wheel & left wheel & right wheel & left wheel \\
\hline
Mean linear deviation $(\delta x, \delta y)$ cm & (0.1, 0.9) & (0.6, 0.8)
& (0.7, 0.9) & (0.2, 0.9) & (0.5, 0.4) & 0.3, 0.4) \\
\hline
Mean ang. deviation ($\delta \theta$) degrees & 0.94 & 2.34 & 2.69 & 1.06 & 0.57 & 0.57 \\
\hline
$\sigma$ in final pose (x,y) cm & (0.3, 0.3) & (0.3, 0.3) & (0.3, 0.2) & (0.4, 0.5) & (0.2, 0.5) & (0.3, 0.5)\\
\hline
uncertainty in pose (x) cm & $(0.1 \pm 0.3)$ & $(0.6 \pm 0.3)$ & $(0.7 \pm 0.3)$ & $(0.2 \pm 0.4)$ & $(0.5 \pm 0.2)$ & $(0.3 \pm 0.3)$ \\
uncertainty in pose (y) cm & $(0.9 \pm 0.3)$ & $(0.8 \pm 0.3)$ & $(0.9 \pm 0.2)$ & $(0.9 \pm 0.5)$ & $(0.4 \pm 0.5)$ & $(0.2 \pm 0.5)$ \\
\hline
\end{tabular}
\caption{Statistical summary of robot motion}
\label{stats}
\end{table}

\item In fig. \ref{left}, the standard deviation in readings of final wheel position for right and left wheels are (0.3, 0.3) cm and (0.3, 0.3) cm respectively.
\item In fig. \ref{right}, the standard deviation in readings of final wheel position for right and left wheels are (0.3, 0.2) cm and (0.4, 0.5) cm respectively.
\item In fig. \ref{straight}, the standard deviation in readings of final wheel positions for right and left wheels are (0.2, 0.5) cm and (0.3, 0.5) cm respectively.
\item This standard deviation is expected, since there is about 0.2 cm uncertainty in marking the end position of the robot as well as the initial jerky motion introduces some noise.
\item The standard deviations for left and right turn motions are very similar.
\item For straight line motion, we observe a substantial variation in y-readings.
\end{itemize}

%%%%%%%%%%%%%%%%%%%%%%%%%%%%%%%%%%%%%%%%%%%%%%%%%%%%%%%%%%%%%%%%%%%%%%%%%%%%%